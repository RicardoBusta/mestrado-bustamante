\section{\textit{EMPTY NAME}}

% ============================================================================
\subsection{Referência completa do artigo}

\begin{itemize}
  \item \textbf{Autores:} EMPTY
  \item \textbf{Local:} EMPTY
  \item \textbf{\textit{Journal}:} EMPTY [Qualis : EMPTY]
  \item \textbf{Data:} EMPTY
  \item \textbf{Referência:} \citeonline{bib:EMPTY}
\end{itemize}


% ============================================================================
\subsection{Resumo}

\subsubsection{Abstract}

\subsubsection{Motivação}

%..........................................................
\subsubsection{Propósito do artigo}

% ..........................................................
\subsubsection{Técnicas utilizadas} 
\begin{itemize}
  \item 
\end{itemize}  

% ..........................................................
\subsubsection{Contribuição em relação a artigos anteriores} %mais ou menos 10 linhas
 \begin{itemize}
   \item 
 \end{itemize}  

% ============================================================================
\subsection{Metodologia}
% Descreva um pouco mais detalhadamente a metodologia e os resultados do artigo. 
% Inclua as figuras que achar mais relevantes.

\subsubsection{xxx}

% ..........................................................
\subsubsection{Resultados}

% ============================================================================
\subsection{Pontos fortes} %no máximo três
\begin{itemize}
  \item
\end{itemize}  

% ============================================================================
\subsection{Limitações} %no máximo três
\begin{itemize}
  \item 
\end{itemize} 


% ============================================================================
\subsection{Avaliação}
%\textbf{(a) Avanço considerável (\textit{Breakthrough}).}
% \textbf{(b) Contribuição significativa.}
% \textbf{(c) Contribuição modesta.}
% \textbf{(d) Contribuição fraca.}
% \textbf{(e) Sem contribuição.}

% ============================================================================
\subsection{Problema em aberto}
 \begin{itemize}
   \item 
 \end{itemize}  

% ============================================================================
\subsection{Aspecto obscuro}
 \begin{itemize}
   \item
 \end{itemize}  