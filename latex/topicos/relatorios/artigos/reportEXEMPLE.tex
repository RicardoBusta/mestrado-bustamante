\section{\textit{xxx}}


% ============================================================================
\subsection{Referência completa do artigo}

\begin{itemize}
  \item \textbf{Autores:} xxx
  \item \textbf{Local:} xxx
  \item \textbf{\textit{Journal}:} xxx [Qualis XXX]
  \item \textbf{Data:} xxx
  \item \textbf{Referência:} \citeonline{xxx}
\end{itemize}


% ============================================================================
\subsection{Resumo}
% ..........................................................
\subsubsection{Propósito do artigo}
xxx

% ..........................................................
\subsubsection{Técnicas utilizadas} 
% \begin{itemize}
%   \item xxx
% \end{itemize}  
xxx

% ..........................................................
\subsubsection{Contribuição em relação a artigos anteriores} %mais ou menos 10 linhas
% \begin{itemize}
%   \item xxx
% \end{itemize}  
xxx

% ============================================================================
\subsection{Metodologia}
% Descreva um pouco mais detalhadamente a metodologia e os resultados do artigo. 
% Inclua as figuras que achar mais relevantes.
xxx

% ..........................................................
\subsubsection{Resultados}
xxx

% ============================================================================
\subsection{Pontos fortes} %no máximo três
\begin{itemize}
  \item xxx
  \item xxx
  \item xxx
\end{itemize}  

% ============================================================================
\subsection{Limitações} %no máximo três
\begin{itemize}
  \item xxx
  \item xxx
  \item xxx
\end{itemize} 


% ============================================================================
\subsection{Avaliação}
\textbf{(a) Avanço considerável (\textit{Breakthrough}).}
% \textbf{(b) Contribuição significativa.}
% \textbf{(c) Contribuição modesta.}
% \textbf{(d) Contribuição fraca.}
% \textbf{(e) Sem contribuição.}
Justificativa.

% ============================================================================
\subsection{Problema em aberto}
% \begin{itemize}
%   \item xxx
% \end{itemize}  


% ============================================================================
\subsection{Aspecto obscuro}
% \begin{itemize}
%   \item xxx
% \end{itemize}  
