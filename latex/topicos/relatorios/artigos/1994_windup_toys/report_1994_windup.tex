\section{\textit{Virtual Wind-up Toys for Animation}}

% ============================================================================
\subsection{Referência completa do artigo}

\begin{itemize}
  \item \textbf{Autores:} Mitchel van de Panne, Ryan Kim, Eugene Fiume
  \item \textbf{Local:} Department of Computer Science and Electrical Engineering, University of Toronto, Canada
  \item \textbf{\textit{Journal}:} xxx [Qualis XXX]
  \item \textbf{Data:} xxx
  \item \textbf{Referência:} \citeonline{bib:1994:winduptoys}
\end{itemize}


% ============================================================================
\subsection{Resumo}
% ..........................................................
\subsubsection{Abstract}
We propose a new method of automatically finding periodic modes of locomotion for arbitrary articulated figures. Cyclic pose control graphics are used as our control representation. These specifically constrain the controller synthesis process to only those controllers producing periodic driving functions. It is shown that stochastic generate-and-test techniques work well with this representation. Several choices that arise in this synthesis technique are explored. The impact of the design of the physical models upon the motions produced is examined. Lastly, the motions produced are analysed by looking at their bifurcation diagrams.
% ..........................................................
\subsubsection{Motivação}
Este artigo apresenta uma tecnica de controladores que permite criação de movimentos complexos e que respeitam as restrições físicas impostas, tentando manter a solução livre de casos específicos.
% ..........................................................
\subsubsection{Propósito do artigo}

Nele é apresentada uma forma de encontrar modos periódicos de locomoção para produzir animações fisicamente realistas para um modelo qualquer.

% ..........................................................
\subsubsection{Técnicas utilizadas} 
 \begin{itemize}
   \item Controladores
   \item Grafo de controle de pose
 \end{itemize}  
xxx

% ..........................................................
\subsubsection{Contribuição em relação a artigos anteriores} %mais ou menos 10 linhas
% \begin{itemize}
%   \item xxx
% \end{itemize}  
xxx

% ============================================================================
\subsection{Metodologia}
% Descreva um pouco mais detalhadamente a metodologia e os resultados do artigo. 
% Inclua as figuras que achar mais relevantes.



% ..........................................................
\subsubsection{Resultados}
asd

% ============================================================================
\subsection{Pontos fortes} %no máximo três
\begin{itemize}
  \item xxx
  \item xxx
  \item xxx
\end{itemize}  

% ============================================================================
\subsection{Limitações} %no máximo três
\begin{itemize}
  \item xxx
  \item xxx
  \item xxx
\end{itemize} 


% ============================================================================
\subsection{Avaliação}
\textbf{(a) Avanço considerável (\textit{Breakthrough}).}
% \textbf{(b) Contribuição significativa.}
% \textbf{(c) Contribuição modesta.}
% \textbf{(d) Contribuição fraca.}
% \textbf{(e) Sem contribuição.}
Justificativa.

% ============================================================================
\subsection{Problema em aberto}
% \begin{itemize}
%   \item xxx
% \end{itemize}  
xxx

% ============================================================================
\subsection{Aspecto obscuro}
% \begin{itemize}
%   \item xxx
% \end{itemize}  
xxx