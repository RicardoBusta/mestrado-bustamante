\section{\textit{SIMBICON: Simple Biped Locomotion Control}}

% ============================================================================
\subsection{Referência completa do artigo}

\begin{itemize}
  \item \textbf{Autores:} KangKang Yin, Kevin Loken, Michiel van de Panne*
  \item \textbf{Local:} University of British Columbia
  \item \textbf{\textit{Journal}:} SIGGRAPH [Qualis : A2]
  \item \textbf{Data:} December 2007
  \item \textbf{Referência:} \citeonline{bib:2007:simbicon}
\end{itemize}


% ============================================================================
\subsection{Resumo}

\subsubsection{Abstract}
Physics-based simulation and control of biped locomotion is diffcult because bipeds are unstable, underactuated, high-dimensional dynamical systems. We develop a simple control strategy that can be used to generate a large variety of gaits and styles in real-time, including walking in all directions (forwards, backwards, sideways, turning), running, skipping, and hopping. Controllers can be authored using a small number of parameters, or their construction can be informed by motion capture data. The controllers are applied to 2D and 3D physically-simulated character models. Their robustness is demonstrated with respect to pushes in all directions, unexpected steps and slopes, and unexpected variations in kinematic and dynamic parameters. Direct transitions between controllers are demonstrated as well as parameterized control of changes in direction and speed. Feedback-error learning is applied to learn predictive torque models, which allows for the low-gain control that typifies many natural motions as well as producing smoother simulated motion.
% ..........................................................
\subsubsection{Propósito do artigo}

% ..........................................................
\subsubsection{Técnicas utilizadas} 
\begin{itemize}
  \item 
\end{itemize}  

% ..........................................................
\subsubsection{Contribuição em relação a artigos anteriores} %mais ou menos 10 linhas
 \begin{itemize}
   \item 
 \end{itemize}  

% ============================================================================
\subsection{Metodologia}
% Descreva um pouco mais detalhadamente a metodologia e os resultados do artigo. 
% Inclua as figuras que achar mais relevantes.

\subsubsection{xxx}

% ..........................................................
\subsubsection{Resultados}

% ============================================================================
\subsection{Pontos fortes} %no máximo três
\begin{itemize}
  \item
\end{itemize}  

% ============================================================================
\subsection{Limitações} %no máximo três
\begin{itemize}
  \item 
\end{itemize} 


% ============================================================================
\subsection{Avaliação}
%\textbf{(a) Avanço considerável (\textit{Breakthrough}).}
 \textbf{(b) Contribuição significativa.}
% \textbf{(c) Contribuição modesta.}
% \textbf{(d) Contribuição fraca.}
% \textbf{(e) Sem contribuição.}

% ============================================================================
\subsection{Problema em aberto}
 \begin{itemize}
   \item 
 \end{itemize}  

% ============================================================================
\subsection{Aspecto obscuro}
 \begin{itemize}
   \item
 \end{itemize}  