% Limpa cabeçalhos.
% (solução para lidar com a númeração das páginas pré-textuais).
\pagestyle{empty}

%% Capa
\begin{titlepage}

% Se quiser uma figura de fundo na capa ative o pacote wallpaper
% e descomente a linha abaixo.
% \ThisCenterWallPaper{0.8}{nomedafigura}

\begin{center}
{\LARGE Ricardo Bustamante de Queiroz}
\par
\vspace{200pt}
{\Huge Título a ser definido}
\par
\vfill
\textbf{{\large Fortaleza}\\
{\large \the\year}}
\end{center}
\end{titlepage}

% Faz com que a página seguinte sempre seja ímpar (insere pg em branco)
\cleardoublepage

% Numeração em elementos pré-textuais é opcional (ativada por padrão).
% Para desativá-la comente a linha abaixo.
\pagestyle{fancy}

% Números das páginas em algarismos romanos
\pagenumbering{roman}

%% Página de Rosto

% Numeração não deve aparecer na página de rosto.
\thispagestyle{empty}

\begin{center}
{\LARGE Ricardo Bustamante de Queiroz}
\par
\vspace{200pt}
{\Huge Título a ser definido}
\end{center}
\par
\vspace{90pt}
\hspace*{175pt}\parbox{7.6cm}{{\large Pré-proposta apresentada ao Departamento de Computação da Universidade Federal do Ceará}}

\par
\vspace{1em}
\hspace*{175pt}\parbox{7.6cm}{{\large Orientador: Creto Vidal}}

\par
\vfill
\begin{center}
\textbf{{\large Fortaleza}\\
{\large \the\year}}
\end{center}

\newpage

% Ficha Catalográfica
\hspace{8em}\fbox{\begin{minipage}{10cm}
Aluno, Ricardo Bustamante de Queiroz.

\hspace{2em} Título a ser definido

\hspace{2em}\pageref{LastPage} páginas

\hspace{2em}Pré-proposta (Mestrado) - Universidade Federal do Ceará. Departamento de Computação.

\begin{enumerate}
\item Animação
\item Realidade Virtual
\item Treinamento de agente virtual
\end{enumerate}
I. Universidade Federal do Ceará. Departamento de Computação.

\end{minipage}}
\par
\vspace{2em}
\begin{center}
{\LARGE\textbf{Comissão Julgadora:}}

\par
\vspace{10em}
\begin{tabular*}{\textwidth}{@{\extracolsep{\fill}}l l}
\rule{16em}{1px} 	& \rule{16em}{1px} \\
Prof. Dr. 		& Prof.ª Dr. \\
asd			& asd
\end{tabular*}

\par
\vspace{10em}

\parbox{16em}{\rule{16em}{1px} \\
Prof. Dr. \\
asd}
\end{center}

\newpage

% Epígrafe
\vspace*{0.4\textheight}
\noindent{\LARGE\textbf{Epígrafe}}
% Tudo que você escreve no verbatim é renderizado literalmente (comandos não são interpretados e os espaços são respeitados)
\begin{verbatim}
Alguma citação interessante.
\end{verbatim}
\begin{flushright}
Quem disse a frase.
\end{flushright}

\newpage

% Agradecimentos

% Espaçamento duplo
\doublespacing

\noindent{\LARGE\textbf{Agradecimentos}}

Texto de agradecimento

\newpage

% Desabilitar protrusão para listas e índice
\microtypesetup{protrusion=false}

% Lista de figuras
\listoffigures

% Abreviações
% Para imprimir as abreviações siga as instruções em 
% http://code.google.com/p/mestre-em-latex/wiki/ListaDeAbreviaturas
%\printnomenclature

% Índice
\tableofcontents

% Re-habilita protrusão novamente
\microtypesetup{protrusion=true}
